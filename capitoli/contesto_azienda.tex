% !TEX encoding = UTF-8
% !TEX TS-program = pdflatex
% !TEX root = ../tesi.tex

%**************************************************************
\chapter{Il contesto aziendale}
\label{cap:processi-metodologie}
%**************************************************************

%**************************************************************
\section{L'azienda: \textbf{\azienda}}
\azienda\ nasce nel 2013 come azienda focalizzata nello sviluppo di soluzioni basate sul comportamento in materia di sicurezza e anti frode. Nel 2014 ha rilasciato le prime soluzioni sulla protezione ed analisi delle transazioni finanziarie.\\
XTN è situata in tre diverse sedi nel nord Italia, Padova, Milano e Rovereto. Attualmente è attiva  sia nello sviluppo di soluzioni \emph{B2B}\glsfirstoccur anti frode che di applicazioni per la protezione di dispositivi mobile e IoT.\\
  
\section{Prodotti offerti}
\azienda\ offre 2 prodotti di punta, \textbf{Smash\textregistered} e \textbf{More\textregistered}.\\
\\
\textbf{Smash\textregistered} è un \emph{framework}\glsfirstoccur\ che analizza il comportamento abituale, grazie a più di cento parametri, degli utenti di servizi di pagamento online. Grazie a questa funzionalità riesce a stabilire, in tempo reale, il fattore di rischio di ogni transazione.\\
Il fattore di rischio è calcolato attraverso svariati algoritmi comportamentali personalizzabili dall'utente tramite un apposito \textit{editor} integrato alla piattaforma. Se, grazie a questi algoritmi, una transazione dovesse risultare sospetta, questa verrà notificata ad una persona incaricata a verificarne l'effettiva natura.\\
Non vengono analizzate solamente i parametri delle transazioni, ad esempio il destinatario o la geocalizzazione, ma anche fattori esterni come anomalie nei protocolli di comunicazione, manipolationi html, compromssioni del dispositivo mobile.\\

Questo prodotto è destinato solamente a istituti finanziari, negozi online e assicurazioni.




\section{Organizzazione aziendale}
Descrizione dell'organizzazione aziendale (divisione in Squads, Chapters)
\section{Processi aziendali}
\subsection{Processo di sviluppo}
\subsection{Gestione della configurazione}
\subsection{Processo di manutenzione}
\subsection{Processo di verifica}
\section{Strumenti a supporto dei processi}
\section{Rapporto con l'innovazione}
Descrizione del loro rapporto con l'innovazione, come vengono individuate le nuove funzionalità dei loro prodotti.

